\section{Discussion}

%summary of method and the problem of loops
We generalized a model of Brownian motion on trees to networks for estimating dispersal rates and locating genetic ancestors from the ARG and developed an algorithm for it. By using the full ARG, our method uses more information than previous methods (Figure \ref{fig:3samARG}) and therefore estimates location of genetic ancestors with less uncertainty (Figure \ref{fig:SingleLineage}), for the same dispersal rate estimate. However, we discovered an interesting but unfortunate property of the model, the dispersal estimate increases monotonically as we use larger genomes with more marginal trees (Fig \ref{fig:DispRate}a). We narrow down the root cause of this, both through simulations and mathematical proofs, to a reduction in the variance of loop node locations (and consequently that of samples) under a BM model (forward in time) as we increase the number of recombination events for the same dispersal rate (Fig \ref{fig:VarToy}, Fig \ref{fig:DispRate}c and appendix \ref{appendix:clusteringproof}). Intuitively, we assume the two lineages precisely meet, which is an unlikely event \citep{Etheridge2019}, especially if they were ever far apart. This implies that the two lineages must not have dispersed very far from one another since their common ancestor, pulling all nodes in and below a recombination loop closer together (Figure \ref{fig:DispRate}c). Therefore, given the same sample locations, the addition of a recombination loop increases our dispersal estimate. As every additional tree comes with an additional recombination loop, our dispersal estimate increases with the number of trees (Figure \ref{fig:DispRate}A). This pulling in of node locations below recombination loops also causes a center-bias in ancestral locations (Figure \ref{fig:ancestral_locations_plots}).

Despite the problem of loops leading to slightly center-biased location estimates (Fig \ref{fig:ancestral_locations_plots}), we can spatially locate recombination nodes well and therefore geographically track a sample's genetic ancestry back in time as it splits and merges via recombination and coalescence. This allows us to locate recombination events of interest and visualize admixed geographic ancestries ({Figure \ref{fig:tracking_recomb_sample}}). This aspect of our model, remains a potential use case even with the problem of loops. For instance, recombination brings advantageous alleles together on the same background, and so identifying where recombinant haplotypes arose can be crucial to understanding the geography of adaptation. For example, new strains of circulating viruses and pathogens are often a consequence of recombination between existing strains, so being able to identify where those recombinant strains first arose is an important step in understanding the spread of pathogens \citep{tamura_virological_2023, ignatieva_ongoing_2022}.

The problem of loops poses a key challenge in utilizing ARGs for spatial inference. Simple models of BM, despite the "pain in torus" \cite{Felsenstein1975}, have continued to be useful and accurate for inference using trees \cite{} (ADD REFERENCES). However, this additional and distinct problem of loops makes BM models, atleast in its simpler forms, an unattractive tool for spatial inference on ARGs. We also explore some simple alternatives to our model. We looked at a model where lineages do not have to meet at recombination nodes (Appendix \ref{appendix:RelaxedBM}), i.e. two parents of a recombination node can be any distance apart, and placed the recombination node at their midpoint \citep[an approach used for phenotypic evolution in phylogenetic networks][]{Bastide2018}. Not forcing the parental lineages to meet greatly decreases the constraint on loop node locations, i.e. increases their (forward in-time) variance, leading to lower dispersal estimates (Figure \ref{fig:S_DispersalRate}). However, the dispersal estimate still increases with the number of trees, due to some constraint that still remains at the recombination nodes (Figure \ref{fig:VarToy} biii). Therefore, to develop an analytically tractable (or atleast computationally feasible) model that can utilize the complete information encoded in an ARG and provide accurate estimates with confidence intervals remains an open problem. From the insights gained from this project, we highlight some ways forward given this problem of loops. 


%We have developed a method for estimating dispersal rates and locating genetic ancestors from ancestral recombination graphs. By using the full ARG for inference, our method can make use of the complete genetic history of the samples. After extending a model of Brownian motion from trees to graphs, we developed an efficient algorithm that scales to large ARGs (Figures \ref{fig:method} and \ref{fig:algo_and_bench}). We then showed that our method, while appropriately handling non-independence across trees, results in biased dispersal estimates, in part due to excess constraint on internal node locations (Figure \ref{fig:DispRate}). When locating ancestors, we show that our method uses more information than previous methods (Figure \ref{fig:3samARG}) and therefore locates with less uncertainty (Figure \ref{fig:SingleLineage}). However, the excess constraint on internal node locations again leads to a bias, but this bias can be reduced by analyzing smaller genomic windows (Figure \ref{fig:ancestral_locations_plots}). An important application of our method is that we can geographically track a sample's genetic ancestry back in time as it splits and merges via recombination and coalescence. This allows us to locate recombination events of interest and visualize admixed geographic ancestries ({Figure \ref{fig:tracking_recomb_sample}}).

%Using information from multiple trees, our method estimates the location of ancestors with more confidence. This may help improve our ability to infer historical migrations, e.g., in humans. Further, appropriately modeling the splitting of recombinant lineages and thereby locating recombination events has multiple potential applications.
%As we have shown, it can provide a more complete picture of the geographic ancestry of admixed individuals (Fig \ref{fig:tracking_recomb_sample}). 

% reasons for biased estimates
%Brownian motion is a convenient but rough approximation for the movement of genetic ancestors down an ARG. For the case of trees, it is well known that the assumptions of constant global population size and independent branching Brownian motions lead to a clustering of individuals \citep{Felsenstein1975,barton2010new}, not the uniform distribution we see in our simulations and expect in nature as a result of local density-dependence. Further, our simple model of Brownian motion assumes an unbounded space. As a result of the latter, dispersal estimates from trees generated in finite space show a systematic downward bias \citep[Figure \ref{fig:DispRate}A and][]{Ianni2022,kalkauskas2021sampling}.
%Modeling Brownian motion down ARGs includes an additional approximation, the meeting of lineages at recombination nodes. Here we assume the two lineages precisely meet, which is an unlikely event \citep{Etheridge2019}, especially if they were ever far apart. This implies that the two lineages must not have dispersed very far from one another since their common ancestor, pulling all nodes in and below a recombination loop closer together (Figure \ref{fig:DispRate}C). This is another way of describing an increased constraint on internal node locations (see Results). Therefore, given the same sample locations, the addition of a recombination loop increases our dispersal estimate. As every additional tree comes with an additional recombination loop, our dispersal estimate increases with the number of trees (Figure \ref{fig:DispRate}A). This pulling in of node locations below recombination loops also explains our bias in ancestor locations (Figures \ref{fig:ancestral_locations_plots} and \ref{fig:LocationErrorBias}).

%ways to reduce bias, part 1
One way to get less biased parameter estimates is to use a more accurate model. An obvious choice is the spatial $\Lambda$-Fleming-Viot (SLFV) process \citep{barton2010new,barton2010newEvol}. This approach models local density dependence and a finite habitat and indeed leads to more accurate estimates than simple Brownian motion when applied on trees \citep{kalkauskas2021sampling}. The SLFV model can be extended naturally to incorporate recombination \citep{barton2010newEvol,etheridge2012spatial} between lineages that are not at the exact same location. One of the key steps in utilizing the SLFV model for inference with ARGs is to extend the model to incorporate recombination along a continuous genome (as opposed to the two loci version of \cite{etheridge2012spatial}) which, while promising, is a difficult problem. Further, the complexity of this model makes it computationally intensive for inference and therefore limited to small sample sizes, even in the case of no recombination \citep{Wirtz2023}. It may then be helpful to think of ways to improve the more tractable and scalable model like Brownian motion. One possibility is to explore other Levy processes to model movement that are not independent to begin with, for instance compound poisson processes which do show up as limiting cases of the SLFV \citep{barton2010new}.
Another option would be to use machine learning algorithms to correct for the model misspecification. The dispersal rate estimate increases quite linearly with the number of marginal trees and the slope of the increases depends on the true dispersal rate, the mating kernel and the competition kernel. These properties of the errors makes it amenable for techniques like domain adaptation (REF)

%Assuming that randomly moving lineages must precisely meet at recombination nodes is excessive, unlikely, and generates a lot of constraint. In our simulations individuals mate with others nearby according to a mating kernel. This could describe, for example, pollen/gamete dispersal or non-random movements between nearby individuals that come together to mate. We therefore explored the alternative assumption, that lineages do not have to precisely meet at recombination nodes (Appendix \ref{appendix:RelaxedBM}). To make this alternative tractable we assumed the opposite extreme, that the two parents of a recombination node can be any distance apart, and placed the recombination node at their midpoint \citep[an approach used for phenotypic evolution in phylogenetic networks][]{Bastide2018}. Not forcing the parental lineages to meet greatly decreases the constraint on internal node locations across trees, leading to lower dispersal estimates (Figure \ref{fig:S_DispersalRate}). However, the dispersal estimate still increases with the number of trees, indicating that some excess constraint remains.


%PARAGRAPH ON CONNECTIONS TO MODELING EVOLUTION ON PHYLOGENETIC NETWORKS.

A connection worth highlighting with modeling trait evolution on phylogenetic networks. It is well known, and easy to see, that models of BM for modeling trait evolution on phylogenetic trees is identical to modeling spatial movement on coalescent trees (REFERENCES). Phylogenetic networks potentially provide a similar structural counterpart to ARGs, where reticulations are due to hybridization, geneflow, or introgression. Unlike in the case of trees, there is an important difference worth keeping in mind while dealing with stochastic processes on networks. In the case of spatial process on a network like ARG, the lineages are expected to meet exactly at the recombination node but when modeling trait evolution on phylogenetic networks, the traits of the the parental populations can arbitrarily far apart when the two populations meet. This is the key different between our model and that of \cite{Bastide2018} which models trait evolution on phylogenetic networks. The methods differ precisely at what happens at the reticulation nodes and we highlight how they are connected via the full paths matrix (see Appendix \ref{appendix:RelaxedBM}). Although this difference is the norm, there are exceptions in both situations (long-range dispersal allows parents to be located far apart, and genetic barriers to gene flow might restrict how different parental population traits can be), which furthers the case for more crosstalk between the fields. . The problem of loops is not a problem for modeling trait evolution on phylogenetic networks given they have the correct phylogenetic network. However, our results imply that their estimate of rate of evolution would be extremely sensitive to the number of reticulations and inferring that number accurately is a major issue in phylogenetic inference which can cause the evolution rate estimates to become biased. In general, modeling stochastic processes on networks is a current focus of both fields and would benefit from more exchange. 

%ways to reduce bias, part 3
%The other alternative we explored to reduce excess constraint on internal node locations was to use only a portion (window) of the available sequence. This approach strikes a compromise between using the full ARG from the entire sequence, which exhibits too much constraint, vs.\ assuming each local tree is spatially independent, which loses information. We show that windowing consequently produces dispersal estimates (Figure \ref{fig:DispRate}) and distributions of ancestor locations (Figure \ref{fig:ancestral_locations_plots}) that fall between those estimated using the full ARG and those assuming independent trees. Given that local trees locate ancestors with less bias than the full ARG (in our simulations) but ignore recombination events, a key advantage of the windowing approach is to locate recombination events and visualize the geographic history of admixed samples (Figure \ref{fig:tracking_recomb_sample}). However, there are some caveats to this approach. Most importantly, it is not possible in practice to know a suitable window size for the parameter you want to estimate. There is also additional computation burden as our algorithm needs to run separately for every window considered (e.g., for ancestors centered on different local trees).


%hurdles to application, part 1
%Here we have developed a model and algorithm for inferring spatial histories from ARGs and tested our method on simulated data but additional issues arise in the application to empirical data. One is that our method is designed to be applied on the full ARG, a single graph with recombination loops. {\tt ARGweaver} \citep{Rasmussen2014} infers full ARGs on which our method can be applied. However, the methods that scale to larger sample sizes and sequence lengths, {\tt tsinfer}+{\tt tsdate} \citep{kelleher2019inferring,Wohns2022} and {\tt Relate} \citep{speidel2019method}, lack some information contained in the full ARG, e.g., they both lack marked recombination nodes. While our method could be immediately applied to a simplified ARG, it is not clear if our model of motion is appropriate (e.g., we would then force lineages to meet not at a recombination node but at the coalescent node below it) or if it could be modified to suit. Scaling-up the inference of full ARGs \citep[e.g.,][]{Deng2024} will allow our method to be applied to larger datasets.

%hurdles to application, part 2
%The main hurdle to applying our method to real datasets, even small ones, is that we have assumed that we know the ARG with complete certainty. This is not possible in practice. To  incorporate uncertainty in ARG inference we could use importance sampling \citep{Osmond2021}, which would require knowing the probability of an ARG under both the assumptions used to infer it and the assumptions of our spatial model. Both {\tt ARGweaver} and {\tt SINGER} provide the probability of each sampled ARG under their panmictic assumptions. The remaining hurdle is to derive the probability of an ARG under our spatial model.

%conclusion
In summary, we have developed a mathematically rigorous model that uses the complete genealogical history of a set of samples to reconstruct the spatial history of their genetic ancestors. We highlight a key issue, the problem of loops, especially in estimating dispersal rates using BM models on ARGs. Finally, we show the utility of the model in visualizing the ancestry of admixed samples and discuss alternative models to deal with the problem of loops. 


% %locating recombination events
% As our method uses information from multiple trees, it can estimate locations of ancestral individuals with more confidence than tree based methods (Fig \ref{fig:SingleLineage}), it can improve inferences about historical migration patterns, e.g., in humans or for invasive species \citep{Guillemaud2010, Hofman2018}. Further, being able to infer the geographic locations of ancestral recombination events has multiple potential applications. It provides a more complete picture of the geographic ancestry of admixed individuals (Fig \ref{fig:tracking_recomb_sample}). Recombination can also be key to bringing advantageous alleles together on the same background, and so identifying where recombinant haplotypes arose can be key to understanding the geography of adaptation. For example, new strains of circulating viruses and pathogens are often a consequence of recombination between existing strains, so being able to identify where those recombinant strains first arose is an important step in understanding the spread of pathogens \citep{tamura_virological_2023, ignatieva_ongoing_2022}.
%reasons for dispersal bias
% With regards to estimating dispersal rates, our method, despite using the full likelihood under a Brownian motion model, show peculiar trends (Fig \ref{fig:DispRate}). Surprisingly, including more trees in the ARG leads to a monotonic increase in the dispersal rate estimates. Under Brownian motion, the meeting of two independent lineages that start at the same point means that the two lineages did not disperse very far from one another, in order for them to meet each other again while moving randomly. This causes the location of a node below a recombination loop to have lower variance. Therefore, given the same sample node locations, the existence of a recombination loop in the ARG increases our dispersal estimate. As every additional tree in the ARG comes with an additional recombination loop, our dispersal estimate increases with the number of trees. \cite{Etheridge2019} hint at a related issue, specifically, the low probability of independent Brownian motions meeting in higher dimensions. In our application, this manifests as an increased dispersal estimate. 

% There may be computational limitations to the size of windows that can be used, as our algorithm must run for each unique window individually. This is not an issue for tracking a specific ancestral lineage, as they all share a single window, but is more prominent when estimating ancestors for many different regions across the chromosome. As some coalescent nodes may appear in many trees in the ARG, these nodes will have multiple location estimates that differ based of the window used. We did not investigate methods for determining the appropriate window size to apply to a given ARG.

%model extensions/alternatives
% Apart from extensions and modifications to the current method that may provide more accurate dispersal rate estimates, there are other interesting movement models that are worth exploring for various use cases. For instance, the most tractable extension would be more general Levy processes. Such extensions are common in phylogenetic inference for tree-like phylogenies but can be adapted to phylogeographic inference with networks. Symmetric Levy processes with jumps can be used when dealing with populations that are tend to have rare bouts of high migration \citep{Landis2013,Dunchen2018}. Directional Levy processes, like Ornstein-Uhlenbeck processes can be used if there resources, like rivers, that guide the dispersal \citep{Cressler2015}. Another extension would be to use asymmetric process, like the $\alpha$-stable Levy process. However, inference with such a process is difficult and, moreover, signals of asymmetric movement should be captured by our method since it should lead to an asymmetric present day distribution. An important modification would be to account for limitations due to topography and habitat boundaries, which ultimately reduces the potential applicability of these methods in certain systems. For a dataset with terrestrial samples spread across multiple continents, should lineage movement be limited to land or is it acceptable to disperse over large bodies of water? Depending on the dispersal mechanism of the system, one event may be more likely than the other, and while our method would be suitable for the later case, where dispersal is not limited by water bodies, more individual-based models would be required for cases when the possible range of dispersal is not continuous \citep{Coulon2008} (OTHER CITATIONS?).

%time varying dispersal
% Currently, our method estimates a single dispersal rate for the entire ARG. However, empirical systems often have temporal and/or spatial heterogenity in dispersal rates due to the environment or behavior of the organisms. To account for temporal variation in dispersal rate, a natural extension of these methods would involve breaking the ARG into epochs, and estimating the dispersal rates of each epoch independently \citep{Osmond2021}. Such a model would often run into identifiability issues, which can be resolved using appropriate priors (see \cite{Lemey2010}).  Spatial heterogeneity, on the other hand, is more difficult to incorporate (but see \cite{grundler2024}). 
