\section*{Cover Letter}

Dear Editor, 

We are excited to submit our manuscript, "Inferring the geographic history of recombinant
lineages using the full ancestral recombination graph", to be considered as an Investigation in \textit{Genetics}. 

In this manuscript, we develop a method that makes full use of the ancestral recombination graph (ARG), which efficiently stores the immense amount of information available in genome sequences, to infer dispersal rates and locate genetic ancestors. The basic mathematical model for dispersal that we use, Brownian motion, has been a common inference tool in both spatial population genetics and phylogeography. However, its use is currently limited to tree-like genealogies but such tree-like genealogies do not accurately represent the genetic ancestry in recombining populations. Hence, we need ARGs, which are becoming more and more feasible to infer. Here, we mathematically generalize the model of Brownian motion to an ancestral recombination graph, allowing us to estimate ancestral locations using the full information available from the complete genealogy. In doing so we outline the challenges and benefits of using Brownian motion in the inference of dispersal. The insights generated by our work complement and further the current understanding of these mathematical challenges and potential solutions, which will be of great value to theoretical population genetic readers of \textit{Genetics}. The method also allows us track the genetic lineages of admixed samples back to their sources and locate key recombination events in the sample's ancestry, which is an exciting prospect to the empirical population genetics community of \textit{Genetics}' readership who wish to track the spatial ancestry from contemporary spatial genetic data.

We hope you agree that this work will be of broad interest the readers of \textit{Genetics} and we eagerly anticipate your response. \\
Sincerely, \\
Puneeth Deraje, James Kitchens, Graham Coop and Matthew Osmond